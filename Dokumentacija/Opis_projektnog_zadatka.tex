\chapter{Opis projektnog zadatka}
		
		\section{Opis problema i cilj projekta}
		U današnjem svijetu nezbrinute i napuštene životinje nisu stran pojam. Procijenjeno je kako u Hrvatskoj gotovo 10.000 životinja nema svoj dom. Nasreću, postoje udruge i ustanove koje napuštenim životinjama pružaju osnovnu njegu i toplinu. Te udruge i skloništa spašavaju ranjene i nezbrinute životinje, no često su im kapaciteti popunjeni, a radnici imaju previše posla. S motivacijom da se s jedne strane pomogne radnicima pri brizi za životinje, a s druge strane potakne građane na angažiranost i udomljavanje životinja, nastala je ideja za aplikaciju „Humanitarni šetači pasa“. Glavni je zadatak aplikacije povezati udruge za životinje s građanima koji imaju želju i vrijeme za šetanje pasa te time povećati izglede udomljavanja pasa i psihološkog efekta dobrobiti socijalizacije za psa i za čovjeka.
		
		
		\section{Funkcionalnosti}
		Pokretanjem aplikacije otvara se naslovna stranica na kojoj je prikazano zaglavlje te popis svih registriranih udruga.
		\newline
		Neregistrirani korisnik ima mogućnost pregleda popisa te pretraživanja udruga po nazivu ili lokaciji. Odabirom pojedine udruge otvaraju se detalji njezina profila. U sklopu profila udruge korisnik može dobiti informacije o psima (slike i opisi, raspoloživost za šetnju u određenom vremenskom periodu (datum i vrijeme) te jesu li psi predodređeni za skupne ili za individualne šetnje). Dodatno, prilikom pregleda profila korisniku je dostupna i rang lista registriranih šetača poredanih s obzirom na broj šetnji, broj pasa te duljinu šetnje koju su odradili u posljednjih mjesec dana. Također, dostupne su mu informacije na kojoj se lokaciji nalazi udruga, statistika o šetnjama svih pasa te mogućnost da se prijavi za šetnju pasa. Ako se korisnik odluči na potonju opciju, ima mogućnost prijave u sustav ili registracije ukoliko još nema račun.
		\newline
		Prilikom registracije korisnik bira želi li se prijaviti kao građanin ili kao udruga.
		\newline
		\newline
		Za registraciju građanina potrebno je unijeti:
		\begin{itemize}
			\item korisničko ime
			\item e-mail
			\item ime i prezime
			\item lozinka
			\item broj mobitela
		\end{itemize}
		
		\noindent
		Za registraciju udruge potrebno je unijeti:
	
		\begin{itemize}
			\item naziv
			\item OIB udruge
			\item korisničko ime
			\item adresu
			\item mjesto
			\item e-mail
			\item lozinku
		\end{itemize}
		
		\noindent
		Registracijom u sustav kao građanin, korisniku se otvara mogućnost prijave za šetnju, pregledavanja osobnih podataka, vlastite statistike šetnji te pregledavanja i preuzimanja vlastitog rasporeda u PDF formatu. On svoju statistiku šetnji može označiti javnom kako bi ona dospjela na rang listu na javnoj stranici.
		\newline
		Udruga registracijom, osim mogućnosti prijave u sustav te pregleda vlastitih osobnih podataka, dobiva i mogućnost mijenjanja profila životinja.
		
		\eject
	