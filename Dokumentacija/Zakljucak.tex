\chapter{Zaključak i budući rad}
		
		 Naš zadatak bio je razviti web aplikaciju (prilagođenu za mobilne uređaje) za humanitarno šetanje pasa koja omogućuje udrugama za životinje da ponude nezbrinute pse za šetnju, a građanima da dobrovoljno šetaju pse uz mogućnost pristupa vlastitom rasporedu šetnji i statistici šetača. Nakon 16 tjedana timskog rada, ostvarili smo zadani cilj, a taj proces odvijao se u dvije faze.
		
		 Prva faza projekta započela je okupljanjem razvojnog tima, dodjelom projektnog zadatka, upoznavanjem sa zahtjevima te njihovim definiranjem i dokumentiranjem. Detaljno definiranje zahtjeva na početku rada na projektu pokazalo se kao velika prednost pri implementaciji te omogućilo lakše planiranje rada i podjelu posla. Definiranje dijagrama obrazaca uporabe, sekvencijskih dijagrama, modela baze podataka i dijagrama razreda pružili su nam jasnu i jednoznačnu ideju ostvarenja funkcionalnosti naše aplikacije.
		 
		 Druga faza projekta velikim dijelom sastojala se od kodiranja i implementacije traženih funkcionalnosti. Članovi su si izabrali zadatke prema osobnim predznanjima i interesima, no svi su se barem s nekom od korištenih tehnologija susreli po prvi puta, što ih je prisililo i potaknulo na samostalno učenje i napredak te međusobno dijeljenje znanja. Dobro definirani zahtjevi na početku spriječili su moguće nesporazume koji u kasnijom fazi implementacije mogu biti vrlo problematični i vremenski skupi. Također, bilo je potrebno dokumentirati dijagrame stanja, aktivnosti, komponenti i razmještaja te napisati upute za korištenje razvijene aplikacije.
		 
		 Za vrijeme rada na projektu, članovi su komunicirali putem Whatsappa, na kojem su dijelili najvažnije informacije i dogovarali termine sastanaka, te putem Discorda, gdje su održavali sastanke, pratili dotadašnji napredak i razrađivali daljnje korake rada. Komunikacija članova tima bila je vrlo kvalitetna te je vladala pozitivna radna atmosfera.
		 
		 Sudjelovanje u ovom projektu bilo je vrlo ugodno i poučno iskustvo te nas je kao buduće inženjere obogatilo ne samo tehničkim znanjima, nego i boljim komunikacijskim vještinama i iskustvom rada u grupi. Iako bi prethodno iskustvo rada na sličnim projektima uvelike olakšalo i ubrzalo razvoj naše aplikacije te konačni proizvod zasigurno učinilo kvalitetnijim, zadovoljni smo postignutim rezultatima i vjerujemo da ćemo mnoga stečena znanja ponijeti sa sobom u buduće projekte koji nas čekaju kroz našu karijeru.